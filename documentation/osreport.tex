\documentclass[12pt, a4paper]{report}
\usepackage{personalstyle}

\setlength{\parskip}{1em}

\begin{document}
    \begin{titlepage}
        \centering
        \Huge{Eggshell}\\
        \Large{an Operating Systems project}\\
        \normalsize{CPS1012}\\[10pt]
        \Large{Andre' Jenkins}\\
        \small{76999M}
    \end{titlepage}

    \tableofcontents

    \clearpage

    \chapter{Structure of the code}
        The code was structured into different directories, so that they may
        be organised according to what they are supposed to achieve.
        \section{Main files}
            The files here are ones which are used in the front layer of the
            eggshell. Here you may find a main C file that uses the eggshell
            functions, the main eggshell \texttt{.c / .h} files which either
            call multiple other functions, or are simple enough to be on the
            front layer, and other files such as the Makefile, any scripts,
            testfiles, and other miscellaneous files.

            \begin{description}
                \item[\textbf{main.c} --- ]
                    The main \texttt{C} file that the executable is retrieved
                    from. Uses libraries such as \texttt{eggshell} and 
                    \texttt{linenoise}.
                \item[\textbf{eggshell.c/h} ---]
                    The \texttt{eggshell library} used by the main file in order
                    to start the eggshell and utilise it. Uses multiple libraries
                    that are all found in the \texttt{src} directory.
                \item[\textbf{Makefile/Makefile-GCC} ---]
                    The Makefile necessary to generate the executable. The current
                    default Makefile uses the \texttt{Clang} compiler, for reasons
                    stated in the \texttt{README.md} file. To use the Makefile that
                    utilises the \texttt{GCC} compiler instead, either change the
                    name of the \texttt{Makefile-GCC} file, or run \texttt{switch.sh}
                \item[\textbf{switch.sh} ---]
                    A script that aids in switching compilers for the makefile.
                    This was written for each switching between \texttt{Clang} and
                    \texttt{GCC}, due to the ease of debugging with \texttt{Clang},
                    and the standard nature of the \texttt{GCC} compiler.
                \item[\textbf{README.md} ---]
                    The \texttt{README} file holds the instructions to compiling the
                    program, as well as a quick summary of the program and its utilities.
                \item[\textbf{testinput.txt} ---]
                    A file used to test the capabilities of the eggshell. 
                    Running \texttt{./eggshell test} will immediately launch
                    the eggshell and run this script, to provide a quick 
                    testing method.
                \item[\textbf{LICENSE, .gitignore \& .yml files} ---]
                    Files that are unimportant to the project itself.
                    These were used for \texttt{git} purposes, as the project 
                    was also uploaded as a \texttt{git} repository.
            \end{description}

        \section{Source files}
            The files found here are the bulk of the code making up the 
            eggshell. In here, every single \texttt{eggshell header library}
            is present, all of them with their own specific and complicated purpose.
            These were seperated from the main \texttt{eggshell} file 
            in order to organise the core of the project from the 
            specific elements making up the project itself.

            \begin{description}
                \item[\bf{variables.c/h} ---]
                    Contains all the functions and structs relating to the
                    variables created and stored by the \tx{eggshell}. 
                    For example, all the \bb{shell} variables can be found
                    here.
                \item[\bf{printer.c/h} ---]
                    The main file dealing with the \tx{print} command.
                \item[\bf{proc\_manager.c/h} ---]
                    The file dealing with the execution of external commands.
                \item[\bf{sig\_handler.c/h} ---]
                    Contains the \tx{signal handler} function used in order 
                    to suspend and interrupt processes. Also contains an 
                    additional function in order to reawaken a suspended
                    process.
                \item[\bf{redirection.c/h} ---]
                    The main file dealing with \tx{input/output} redirection.
                \item[\bf{pipe\_manager.c/h} ---]
                    Contains functions dealing with the piping system that the
                    \tx{eggshell} offers. Also contains a special execution function,
                    rather than using the one found in \tx{proc\_manager.c/h}
            \end{description}
        
        \section{Other files}
            There are also other directories that contain files that aren't
            integral to the functionality of the eggshell, but are still related
            somewhat.

            \begin{description}
                \item[\bf{documentation/} ---]
                    Contains the \tx{.tex} file that generated this report,
                    as well as other items related to it. In order to recompile it,
                    you'd most likely need to install \TeX{}live first.
                \item[\bf{ci/} ---]
                    Unrelated to the main project. The Makefile here is used for 
                    Continuous Integration for the \tx{git} repository.
                \item[\bf{add-on/} ---]
                    Contains the \tx{linenoise.c/h} library that was used in the 
                    main file to simulate a terminal's prompt with input.
                \item[\bf{.vscode/} ---]
                    Contains files that helped with debugging/building the project
                    in \tx{Visual Studio Code}.
            \end{description}




\end{document}